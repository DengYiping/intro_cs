\documentclass{article}
\usepackage{amsmath}
\usepackage{amsfonts}
\title{Intro to Computer Science Assignment 3}
\date{2017-09-27}
\author{Yiping Deng}
\begin{document}
\maketitle
\paragraph{Problem 3.1}
Lemma: The number of elements in the power set $P(s)$ of a finite set S with n elements is $2^n$ \\
Proof:\\
\subparagraph{Basis:} Given a empty set $A = \emptyset $, its power set $P(A) = \{ \emptyset \}$
\subparagraph{Inductive proof:} Assume Lemma holds for set $A$ with k element
$\implies $the size of its power set $P(A)$ is $ 2^k \implies $
for set with a extra element $m$, $ B = \{ m \} \cup A $ has $ k + 1 $ elments $\implies $
in its power set $P(B)$, it includes all the element in $P(A) $ and with a a set with its element set unioned with element $m$
$ \implies P(B) = P(A) \cup \{ C \cup \{ m \} \mid C \in P(A) \} \implies$ the size of $P(B)$ is twice the size of $P(A) \implies$ the size of $P(B)$ is equal to $2^{k + 1}$

\paragraph{Problem 3.2}
\subparagraph{a)} \begin{itemize}
\item It is not reflexive: $ a = b \implies \neg a \neq b $
\item It is symmetric: $ a = b \implies b = a $
\item It is transitive: $ a = b \land b = c \implies a = c $
\end{itemize}
\subparagraph{b)} \begin{itemize}
\item It is reflexive: $ \mid a - a \mid = 0 \leq 3 \implies \forall a \in A.(a, a) \in R$
\item It is symmetric: $ \mid a - b \mid \leq 3 \iff \mid b - a \mid \leq 3$
\item It is not transitive: $ \mid a - b \mid \leq 3 \land \mid b - c \mid \leq 3 \not \implies \mid a - c \mid \leq 3 $ .
A example:
$ a = 3, b = 6, c = 9 \implies \mid a - b \mid \leq 3 , \mid b - c \mid \leq 3 , \mid a - c \mid \geq 3 $
\end{itemize}
\subparagraph{c)} \begin{itemize}
\item It is reflexive: $ \forall a \in \mathbb{Z} .(a \, mod \, 10) = (b \, mod \, 10) $
\item It is symmetric: $ \forall a,b \in \mathbb{Z}.(a \, mod \, 10) = (b \, mod \, 10) \\ \implies (b \, mod \, 10) = (a \, mod \, 10) \implies \forall (a,b) \in R, (b,a) \in R$
    \paragraph{Problem 3.3} in problem3.hs file
    \paragraph{Problem 3.4} in problem4.hs file (in the comment)

\end{itemize}
\end{document}
