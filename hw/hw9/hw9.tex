\documentclass{article}
\usepackage{amsmath}
\usepackage{amsfonts}
\usepackage{minted}
\title{Intro to Computer Science Assignment 9}
\date{2017-11-16}
\author{Yiping Deng}
\begin{document}
\maketitle
\paragraph{9.1}
\subparagraph{a)}
\begin{center}
    \begin{tabular}{|l|r|r|r|}
        \hline
        \# & Machine Code & Assembly Code & Description \\
        \hline
        0 & 001 1 0001 & LOAD \#0001 & Load memory at 0001 into the acc \\
        1 & 010 0 1111 & STORE 1111 & Store the acc value to memory 1111 \\
        2 & 001 1 0000 & LOAD \#0000 & Load 0000 into the acc \\
        3 & 101 1 0100 & EQUAL \#0100 & If 0100 is equal to the acc, skip next instruction \\
        4 & 110 1 0110 & JUMP \#0110 & Jump to 0110 \\
        5 & 111 1 0000 & HALT \#0000 & Stop the program, \\
        6 & 001 0 0011 & LOAD 0011 & Load memory location 0011 into acc \\
        7 & 100 1 0001 & SUB \#0001 & Substract 0001 from the acc \\
        8 & 010 0 0011 & STORE 0011 & Store acc to memory location 0011 \\
        9 & 001 0 1111 & LOAD 1111 & Load memory at 1111 to acc \\
        10 & 011 0 1111 & ADD 1111 & Add memory at 1111 to acc \\
        11 & 010 0 1111 & STORE 1111 & Store acc to memory at 1111 \\
        12 & 110 1 0010 & JUMP \#0010 & Jump back to line 2 \\
        13 & 000 0 0000 & NOP & Nothing happens \\
        14 & 000 0 0000 & NOP & Nothing happens \\
        15 & 000 0 0000 & NOP & Nothing happens\\
        \hline
    \end{tabular}
\end{center}
\subparagraph{b)}
\begin{enumerate}
    \item Copy the value from memory 1 to memory 15
    \item if the value at memory 3(the constant 0100, or 4) is is equal to 0, jump to line 6, elsewise stop the program
    \item substact memory at 3 by 1
    \item double the memory at 15
    \item stop the program
\end{enumerate}
\subparagraph{c)}
the initial value of memory block 1 is binary 0001 = decimal 1, it is copied into memory 15, and later memory block 15 is doubled to 2. \\
then you jump back and doubled again, for 3 times. (memory block is decresed by 1 every iteration). \\
When it hits 0, it will terminate. \\
It means that it will be executed 4 times, in theory, you will have 1*2*2*2*2 = 16 \\
It means it will overflow, eventually you will get 0 \\
\paragraph{9.2}
\subparagraph{a)}
RAX, RBX, RCX, RDX, RBP, RSI, RDI, and RSP, R8, R9, R10, R11, R12, R13, R14, R15 are all the available register in x64 architecture \\
RAX is a 64-bit register\\
EAX is a 32-bit register, mapping to the lower 32-bit in RAX\\
AX is a 16-bit register, mapping to the lower 16-bit in EAX\\
AL is a 8-bit register, mapping to the lower 8-bit in AX\\
AH is a 8-bit register, mapping to the higher 8-bit in AX(bit 8 to bit 15)\\
\subparagraph{b)}
The common addressing mode in x64 system is
\begin{enumerate}
    \item Immediate: the value is stored in the instruction.
    \item Register to register: the operand is a memory address.
    \item Indirect: by 8-bit, 16-bit, 32-bit displacement along any general register for base and index.
        Can be multiplied by a factor of 1,2,4,8.
    \item RIP-relative addressing: use a address relative to RIP register.
\end{enumerate}
The stack pointer, RSP, RBP, is maintained in the register(in CPU), RSP pointer points to the stack top, and RBP pointer points to the stack base. \\
\subparagraph{c)}
\inputminted{nasm}{code.asm}
The assembly code simply add nums from 1 to 10, and return.
It does the following job:
\inputminted{c}{code.c}
Eventually, it will return the result in \%rax
\end{document}
