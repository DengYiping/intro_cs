\documentclass{article}
\usepackage[utf8]{inputenc}
\usepackage{amsmath}
\usepackage{amsfonts}
\usepackage{amssymb}
\usepackage{minted}
\usepackage{titlesec}
\usepackage[a4paper,margin=1in,footskip=0.25in]{geometry}
\usepackage{fancyhdr}
\pagestyle{fancy}
%basic page layout

%draw finite state machine
\usepackage{tikz}
\usetikzlibrary{arrows,automata}

%header and footer settings
\lhead{Intro CS Assignment 11}
\chead{Yiping Deng}
\rhead{\today}

\titlelabel{\thetitle\enspace}

\begin{document}
\title{Computer Science Assignment 11}
\author{Yiping Deng}
\maketitle
\thispagestyle{fancy}
\section*{Problem 11.1}
\subsection*{1)}
\begin{align*}
    \delta &= \{((S, a), S1), ((S, b), S0), \\
    &((S0, b), S0), ((S0, a), S1),\\
    &((S1, a), S2), ((S1, b), S0) \\
    &((S2, a), S2), ((S2, b), S2)\} \\
    S &= \{S0, S1, S2, S3\} \\
    s0 &= S \\
    F &= \{S0\} \\
\end{align*}
\subsection*{2)}
\begin{figure}[h]
    \centering
    \begin{tikzpicture}[>=stealth',shorten >=1pt,auto,node distance=4cm]
    \node[initial,state] (S)      {$S$};
    \node[state,accepting] (S0) [right of=S]             {$S0$};
    \node[state]         (S1) [below right of=S0]  {$S1$};
    \node[state]         (S2) [right of=S1] {$S2$};


        \path[->] (S) edge [bend right]      node {a} (S1)
                      edge      node {b} (S0)
        (S0)  edge [loop above] node {b} (S0)
                edge              node {a} (S1)
            (S1) edge [bend left]  node {b} (S0)
                edge              node {a} (S2)
                (S2) edge [loop above] node {a} (S2)
                (S2) edge [loop below] node {b} (S2);
    \end{tikzpicture}
\end{figure}
As shown in the figure, S is the inital state, S0 is the accepting state.
\subsection*{3)}
The following Haskell code will represent the finite state machine above.
\inputminted{Haskell}{code.hs}
\subsection*{4)}
The regular grammar that generates the patterns is described below.
\begin{align*}
    G &= (N, \Sigma, P, S) \\
    N &= \{S, T, U \} \\
    \Sigma &= \{a, b\} \\
    P &= \{ S \mapsto aT, S \mapsto bU, \\
    &U \mapsto \epsilon, U \mapsto bU, \\
    &U \mapsto aT, \\
    &T \mapsto bU \}
\end{align*}

\section*{Problem 11.2}
\subsection*{a)}
The following is the mathematical definition of increment Turing machine.
\begin{align*}
    s0 &= S0 \\
    \Gamma &= \{ 0, 1, \$ \} \\
    b &= \_ \\
    F &= \{ \} \\
    \delta &= \{ (S0, \$, S1, \$, R), \\
    &(S1, \$, S2, \$, L), (S1, 0, S1, 0, R), (S1, 1, S1, 1, R), \\
    &(S2, 0, S3, 1, L), (S2, 1, S2, 0, L), \\
    &(S3, 0, S3, 0, L), (S3, 1, S3, 1, L) \} \\
\end{align*}
S0 is the initial state, just skip a \$ symbol. \\
S1 is the skip state, skipping until hit the last \$ symbol. \\
S2 is the carried state, do the carried addition. \\
S3 is the non-carried state.

The following is the Haskell code validates the Turing machine.
\inputminted{Haskell}{turing_inc.hs}

\subsection*{b}
Mathematical definition of decrement Turing machine 
\begin{align*}
    s0 &= S0 \\
    \Gamma &= \{ 0, 1, \$ \} \\
    b &= \_ \\
    F &= \{ \} \\
    \delta &= \{ (S0, \$, S1, \$, R), \\
    &(S1, \$, S2, \$, L), (S1, 0, S1, 1, R), (S1, 1, S1, 0, R), \\
    &(S2, 0, S3, 1, L), (S2, 1, S2, 0, L), \\
    &(S3, 0, S3, 0, L), (S3, 1, S3, 1, L) \\
    &(S4, 1, S4, 0, R), (S4, 0, S4, 1, R) \} \\
\end{align*}
S0 is the inital state, just skip a \$ symbol. \\
S1 is the right flip state, flip the bits until hit \$. \\
S2 is the carried state, do the carried addition. \\
S3 is the non-carried state.
S4 is the flip state, flip the bits back until hit \$. \\

This is the validating Haskell code. 
\inputminted{Haskell}{turing_dec.hs}

\subsection*{c)}
\end{document}
