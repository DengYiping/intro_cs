\documentclass{article}
\usepackage{amsmath}
\usepackage{amsfonts}
\usepackage{minted}
\title{Intro to Computer Science Assignment 10}
\date{2017-11-23}
\author{Yiping Deng}
\begin{document}
\maketitle
\paragraph{1.}
\subparagraph{a)} $\circ$ is associatetive, $e$ is neutral element $\implies (x \circ y) \circ z = x \circ (y \circ z)$ \\
$x \circ e = x$ \\
using this property, we can prove the inductive basis\\
\begin{minted}{Haskell}
    foldl op e [] = e
    foldr op e [] = e
    foldl op e [] = foldr op e []
\end{minted}
inductive step:\\
\begin{minted}{Haskell}
    foldl op e (x:xs) = (op e x) `op` (foldl op e xs) = x `op` (foldl op e xs)
    foldr op e (x:xs) = x `op` (foldr op e xs)
    -- by inductive hypothesis
    foldl op e xs = foldr op e xs
    -- which implies
    foldl op e (x:xs) = x `op` (foldl op e xs) = x `op` (foldr op e xs) = foldr op e (x:xs)
\end{minted}

\subparagraph{b)} proof: \\
using induction, first we need to prove the basis: \\
\begin{minted}{Haskell}
    foldr op1 e [] = e = foldl op2 e []
\end{minted}
inductive step: \\
\begin{minted}{Haskell}
    -- assumption
    foldr op1 e xs = foldl op2 e xs
    -- proof
    foldl op2 e (x:xs) = foldl op2 (e `op2` x) xs
    = foldl op2 (x `op1` e) xs
    -- using the first rule
    = x `op1` (foldl op2 e xs)
    -- by induction hypothesis
    = x `op1` (foldr op1 e xs)
    -- put it inside of foldr
    = foldr op1 e (x:xs)
\end{minted}
\subparagraph{c)} proof: \\
inductive basis:\\
\begin{minted}{Haskell}
    foldr op a [] = a = foldl op` a [] = foldl op` a (reverse [])
\end{minted}
inductive step:
\begin{minted}{Haskell}
    foldl op` a (reverse (x:xs)) =
    -- since it is reversed
    = (foldl op` a (reverse xs)) `op`` x
    -- by inductive hypothesis
    = (foldr op a xs) `op`` x
    -- by using the rule
    = x `op` (foldr op a xs)
    -- by moving inside
    = foldr op a (x:xs)
\end{minted}

\paragraph{2.}
\subparagraph{a)}
For invocation "./happy-fork", there is no extra argument
$\implies argc == 1$ (the first argument refers to itself)
$\implies$ the main process never goes into the for loop
$\implies$ there will be 0 child process(0 fork invocation) \\
\\
For invocation "./happy-fork a", there is 1 extra argument
$\implies argc == 2$
$\implies$ the main process will be in the loop once
$\implies$ the direct child of main will never call the loop again,
and the direct child will call fork() one more time
$\implies$ there will be 2 child processes\\
\\
For invocation "./happy-fork a b", there is 2 extra argument
$\implies argc == 3$.
The first two fork will have the exact situation as the main process in $argc == 2$, which will have 3 process(including the main).
The main will also behaves like program $argc == 2$.
In total, $(2 + 1) * 3 = 9$ process, which means 8 subprocess. \\
\\
We can easily conclude that every time the program will split into 3 process that behaves like the argc is $argc - 1$. \\
Hence, $ \text{number of subprocess} = 3^{argc - 1} - 1$ \\
Therefore, "./happy-fork a b c" will have 26, and "./happy-fork a b c d" will have 80. \\

This conclusion matched the subprocess counter I created.
\inputminted{c}{code.c}

This program uses pipes to keep track of number of subprocess.
This matches my conclusion perfectly.
\subparagraph{2)}
This is a program that does the same thing as the previous, short C program, without stdlib, only system call.
\inputminted{asm}{l.s}

\end{document}
